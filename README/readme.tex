\documentclass{article}

\usepackage{amsmath}
\usepackage{hyperref}
\usepackage{graphicx}
\usepackage{braket}

\title{Metodi Computazionali della Fisica\\Università degli Studi di Padova}
\author{Ismaele Lorenzon}
\date{\today}
\begin{document}
\maketitle

\section{Introduzione}
Il progetto si pone come base della prova orale per l'esame di Metodi Computazionali della Fisica, facente rifarimento al dipartimento e in particolare al corso 
triennale in fisica presso l'Università degli Studi di Padova.
\subsection{Consegna}
\begin{center}
\begin{quote}
7. Equazione di Schroedinger tempo-indipendente 2D uso dell’algoritmo di 
minimizzazione come  CG (gradiente coniugato) implementati in \texttt{scipy} per trovare lo 
stato fondamentale e i primi stati eccitati. Consideriamo un potenziale a gradino e 
condizioni al contorno periodiche. 
\end{quote}
\end{center}
\subsection{Obiettivo}
L'obiettivo evidente è la risoluzione dell'equazione degli autovalori per l'Hamiltoniana, nel contesto dell'equazione di Schroedinger bidimensionale. La consegna non 
lo rende esplicito ma si rende noto vengano richiesti sia gli autostati che gli autovalori.
\begin{equation}
    \hat{H}\ket{\psi}=E\ket{\psi}
\end{equation}
Dove con $\psi$ indichiamo l'autostato e con $E$ l'energia ad esso associato.
\section{Implementazione Numerica dell'Hamiltoniana}
L'Hamiltoniana associata al problema in questione, è generalmente:
\begin{equation}
    \hat{H} = -\frac{\hbar^2}{2m} \nabla^2 + V(x,y) = -\frac{\hbar^2}{2m} \left(\frac{\partial^2}{\partial x^2} + \frac{\partial^2}{\partial y^2}\right) + V(x,y)
\end{equation}
Il problema viene poi numericamente semplificato tramite la scelta di costanti naturali (o atomiche) adimensionali cioè
\begin{equation}
    \hbar=1 \hspace{10pt} m=1
\end{equation}
da cui le energie risultanti associate agli autostati saranno in Hartree ($\approx 27.2eV$). \\ Considereremo un dominio rettangolare, di dimensioni 
$L_x,L_y$ e discretizzato lungo ciascuna delle dimensioni in $N_x,N_y$ segmenti di lunghezza $h_x,h_y$. Indicheremo poi con $\psi_{i,j}$ l'elemento dell'autostato 
rappresentato da un vettore di lunghezza $N_x * N_y$, atto ad "appiattire" il dominio, corrispondente alle coordinate $x=i*h_x,y=j*h_y$.
\end{document}