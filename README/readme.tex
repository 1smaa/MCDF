\documentclass{article}

\usepackage{amsmath}
\usepackage{hyperref}
\usepackage{graphicx}
\usepackage{braket}
\usepackage{listings}
\usepackage{xcolor}

\lstset{
  language=Python,
  basicstyle=\ttfamily\small,
  keywordstyle=\color{blue},
  stringstyle=\color{orange},
  commentstyle=\color{gray},
  showstringspaces=false,
  breaklines=true,
  frame=single,
  mathescape=true,
  literate={_}{}{0\discretionary{}{}{}} ,
}

\title{Metodi Computazionali della Fisica\\Università degli Studi di Padova}
\author{Ismaele Lorenzon}
\date{\today}
\begin{document}
\maketitle

\section{Introduzione}
Il progetto rappresenta la base della prova orale per l'esame di Metodi Computazionali della Fisica, riferito al Dipartimento di Fisica e in particolare al corso di laurea triennale in Fisica presso l'Università degli Studi di Padova.

\subsection{Consegna}
\begin{center}
\begin{quote}
7. Equazione di Schrödinger tempo-indipendente 2D: utilizzo di un algoritmo di 
minimizzazione, come il CG (gradiente coniugato), implementato in \texttt{scipy}, per trovare lo 
stato fondamentale e i primi stati eccitati. Si consideri un potenziale a gradino e 
condizioni al contorno periodiche. 
\end{quote}
\end{center}

\subsection{Obiettivo}
L'obiettivo è risolvere l'equazione agli autovalori per l'Hamiltoniana, nel contesto dell’equazione di Schrödinger bidimensionale. Sebbene la consegna non lo specifichi esplicitamente, è richiesto il calcolo sia degli autostati che degli autovalori:
\begin{equation}
    H\psi=E\psi
\end{equation}
dove $\psi$ rappresenta l'autostato ed $E$ l'energia corrispondente.

\section{Implementazione Numerica dell'Hamiltoniana}
L’Hamiltoniana del problema in esame è generalmente espressa come:
\begin{equation}
    H = -\frac{\hbar^2}{2m} \nabla^2 + V(x,y) = -\frac{\hbar^2}{2m} \left(\frac{\partial^2}{\partial x^2} + \frac{\partial^2}{\partial y^2}\right) + V(x,y)
\end{equation}
Il problema viene numericamente semplificato scegliendo costanti adimensionali (sistema di unità naturali o atomiche), ovvero:
\begin{equation}
    \hbar=1 \hspace{10pt} m=1
\end{equation}
Le energie risultanti saranno quindi espresse in Hartree ($\approx 27.2\,\text{eV}$).

\subsection{Discretizzazione}
Consideriamo un dominio rettangolare di dimensioni $L_x, L_y$, discretizzato lungo ciascuna delle due dimensioni in $N_x, N_y$ segmenti, di lunghezza $h_x, h_y$ rispettivamente. Indicheremo con $\psi_{i,j}$ l’elemento dell’autostato, rappresentato da un vettore di lunghezza $N_x \cdot N_y$, che “appiattisce” il dominio bidimensionale. L’indice $l$ è definito come $l = x + N_x \cdot y$ per indicizzare correttamente la griglia bidimensionale. \\

Approssimiamo le derivate parziali nel modo seguente:
\begin{equation}
\frac{\partial^2 \psi_{i,j}}{\partial x^2}=\frac{\psi_{i-1,j}-2\psi_{i,j}+\psi_{i+1,j}}{h_x^2}
\end{equation}
e analogamente per la derivata seconda rispetto a $y$.

Possiamo quindi riformulare $H\psi$ come:
\begin{equation}
    H\psi_{i,j} = -\frac{\psi_{i-1,j} - 2\psi_{i,j} + \psi_{i+1,j}}{2h_x^2}
                 - \frac{\psi_{i,j-1} - 2\psi_{i,j} + \psi_{i,j+1}}{2h_y^2}
                 + V_{i,j} \psi_{i,j}
\end{equation}
che può essere riscritto facilmente in forma matriciale. \\

L’implementazione del potenziale è banale e viene dunque tralasciata. L’adattamento a un potenziale generico può essere effettuato tramite una funzione che, date le coordinate, restituisce il valore del potenziale nel punto. \\

Vengono imposte condizioni al contorno periodiche, implementate facilmente tramite l’operatore modulo, che consente di accedere agli elementi della matrice "rientrando" dal lato opposto del dominio. \\

Da tutte queste osservazioni si ricava che l’operatore Hamiltoniano è dato da:

\begin{lstlisting}
def H(nx: int,ny: int,hx: float,hy:float, V: Callable) -> np.ndarray:
    size=nx*ny # Dimensione del vettore appiattito
    hamil=np.zeros(shape=(size,size)) # Inizializzazione della matrice
    hsqx=hx**2 # Precalcolo per maggiore efficienza
    hsqy=hy**2
    for i in range(0,size):
        hamil[i][i]=(1/hsqx+1/hsqy+V(*Operators.l_to_coord(nx,ny,i),hx,hy)) # Diagonale: potenziale in funzione della posizione
        hamil[i][(i-1)%size]=-1/(2*hsqx) #Condizioni periodiche sulle x
        hamil[i][(i+1)%size]=-1/(2*hsqx)
        hamil[i][(i+nx)%size]=-1/(2*hsqy) #Condizioni periodiche sulle y
        hamil[i][(i-nx)%size]=-1/(2*hsqy)
    return hamil
\end{lstlisting}
\subsection{Problema degli autovalori}
Sono possibili diversi approcci alla risoluzione del problema degli autovalori, in questo caso è richiesta la risoluzione tramite minimizzazione del 
funzionale di energia 
\begin{equation}
    E(\ket{\psi})=\bra{\psi} H \ket{\psi} - \lambda(\langle \psi|\psi \rangle-1)
\end{equation}
Utilizzando il moltiplicatore di Lagrange $\lambda$ come vincolo sulla normalizzazione della funzione d'onda. 
Dalle condizioni di minimo $\frac{\partial E(\ket{\psi})}{\partial \bra{\psi}}=0=H\ket{\psi}-\lambda\ket{\psi}$ e $\frac{\partial E(\ket{\psi})}{\partial \lambda}=0=\langle \psi | \psi \rangle -1$ otteniamo la soluzione al nostro problema.
\\ Il progetto contiene due diverse implementazioni di 
questo funzionale:
\begin{itemize}
    \item \texttt{main\_laplacian.py}: minimizzazione effettuata con il funzionale completo anche del moltiplicatore di Lagrange
    \item \texttt{main.py}: normalizzazione costretta all'interno della procedura di ortonormalizzazione di Gram-Schmidt, sperimentalmente più veloce e risultati identici, ma 
matematicamente meno completa.
\end{itemize}
L'algoritmo di minimizzazione richiesto è il metodo del gradiente coniugato. Si tratta di un metodo di discesa del gradiente più costoso dal punto di vista computazionale rispetto al metodo standard dello steepest descent, ma che consente una convergenza più efficiente.
Permette di generare le direzioni di ricerca in maniera tale che la minimizzazione lungo la direzione successiva non rovini le minimizzazioni lungo le direzioni precedenti.
In questo caso è stata richiesto l'utilizzo dell'implementazione della libreria \texttt{scipy}.
\end{document}
